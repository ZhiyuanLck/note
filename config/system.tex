\section{系统配置}

\subsection{代理设置}
\subsubsection{socks转http}
安装 |proxychains4|
\begin{mysh}[|/etc/privoxy/config|]
  socks5 127.0.0.1 1080
\end{mysh}

\subsection{登录界面位置设置}
ubuntu设置:
\begin{mysh}
sudo cp ~/.config/monitors.xml ~gdm/.config/monitors.xml
sudo chown gdm:gdm ~gdm/.config/monitors.xml
\end{mysh}

\subsection{多显示器设置}
安装 |arandr| 将布局保存,每次启动载入布局

\subsection{字体配置}
\subsubsection*{字体下载}
\href{https://github.com/adobe-fonts/source-han-sans/raw/release/OTF/SourceHanSansSC.zip}
{思源黑体}、
\href{https://github.com/adobe-fonts/source-han-serif/raw/release/OTF/SourceHanSerifSC_EL-M.zip}
{思源宋体1}、
\href{https://github.com/adobe-fonts/source-han-serif/raw/release/OTF/SourceHanSerifSC_SB-H.zip}
{思源宋体2}

给polybar用的\href{https://unifoundry.com/pub/unifont/unifont-13.0.05/font-builds/unifont-13.0.05.ttf}
{unifont}和\href{https://unifoundry.com/pub/unifont/unifont-13.0.05/font-builds/unifont-13.0.05.ttf}
{siji}

\subsubsection*{字体优先级设置}
修改文件 |/etc/fonts/conf.avail/64-language-selector-prefer.conf|
